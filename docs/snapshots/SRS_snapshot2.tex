\documentclass[12pt]{article}
\usepackage[margin=1in]{geometry}
\usepackage{longtable}
\usepackage{hyperref}

\title{Software Requirements Specification (SRS)\\
AI Music Recommender System\\
Snapshot 2 - Mood-Based Recommendations}
\author{Project Group 5}
\date{\today}

\begin{document}
\maketitle

\section*{Version Description}
\begin{longtable}{p{2cm}p{9cm}p{3cm}}
\textbf{Version} & \textbf{Description} & \textbf{Date}\\ \hline
2.0 & SRS updated for Snapshot 2. Adds mood-based recommendation feature and corresponding requirements. & \today\\
\end{longtable}

\section{Introduction}
\subsection{Purpose}
This SRS extends Snapshot 1 by specifying requirements for mood-based recommendations, allowing users to select a mood and receive playlists tailored to that mood using predefined or learned song features.[web:16][web:18]

\subsection{Scope}
Snapshot 2 adds mood selection to the UI, mood-aware recommendation logic, and storage of user mood choices, while preserving all Snapshot 1 functionality.

\subsection{Intended Audience}
The audience includes the course instructor, teaching assistants, and project team members implementing and testing the mood-based feature.

\section{New and Updated Requirements}
\subsection{User Interface Extensions}
\begin{itemize}
  \item FR-M1: The system shall provide a mood selector on the home screen (e.g., Happy, Chill, Focus, Energetic).
  \item FR-M2: The system shall display a “Mood Playlist” section showing songs tailored to the selected mood.
\end{itemize}

\subsection{Mood-Based Recommendation}
\begin{itemize}
  \item FR-M3: The system shall allow a user to change the current mood at any time from the home screen.
  \item FR-M4: When the user selects a mood, the system shall generate a playlist of songs associated with that mood using stored mood features or tags.[web:13][web:18]
  \item FR-M5: The system shall use both mood and the user’s interaction history when ordering songs in the mood playlist.
\end{itemize}

\subsection{Interaction and Data Storage}
\begin{itemize}
  \item FR-M6: The system shall store the user’s most recent mood selections for future recommendation tuning.
  \item FR-M7: The system shall log plays, likes, and skips that occur while a mood is active, including the active mood label.
\end{itemize}

\subsection{Search and Navigation}
\begin{itemize}
  \item FR-M8: The user shall be able to navigate between the regular recommendation list and the mood playlist.
  \item FR-M9: The user shall still be able to use search independent of mood selection.
\end{itemize}

\section{Non-Functional Requirements (Snapshot 2)}
\subsection{Performance}
\begin{itemize}
  \item NFR-M1: Updating the mood playlist after changing the selected mood should complete within a few seconds under normal conditions.
\end{itemize}

\subsection{Usability}
\begin{itemize}
  \item NFR-M2: The mood selector shall be clearly visible and easy to understand (e.g., labeled buttons or icons).
\end{itemize}

\section{Legal and Ethical Considerations}
Mood data is treated like other interaction data and is used only to personalize recommendations, following the same privacy principles defined for Snapshot 1 and informing users that mood selections are stored for personalization.[web:16]

\end{document}
