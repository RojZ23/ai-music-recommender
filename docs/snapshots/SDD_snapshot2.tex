\documentclass[12pt]{article}
\usepackage[margin=1in]{geometry}
\usepackage{longtable}
\usepackage{hyperref}

\title{Software Design Document (SDD)\\
AI Music Recommender System\\
Snapshot 2 - Mood-Based Recommendations}
\author{Project Group 5}
\date{\today}

\begin{document}
\maketitle

\section*{Version Description}
\begin{longtable}{p{2cm}p{9cm}p{3cm}}
\textbf{Version} & \textbf{Description} & \textbf{Date}\\ \hline
2.0 & SDD updated for Snapshot 2. Extends architecture and database design for mood-based recommendations. & \today\\
\end{longtable}

\section{Introduction}
\subsection{Purpose}
This document updates the Snapshot 1 design to incorporate mood-based recommendations, including new UI elements, backend logic, and database structures.[web:16][web:18]

\subsection{Scope}
Snapshot 2 modifies the recommendation workflow to accept a mood parameter, adds storage for mood preferences, and updates the client to support mood selection.

\section{Architecture Changes}
\subsection{Updated Workflow}
\begin{enumerate}
  \item The user logs in and views the home screen.
  \item The user selects a mood (e.g., Happy, Chill, Focus).
  \item The client calls the recommendation API with the user identifier and selected mood.
  \item The backend queries song features or tags associated with that mood and the user’s interaction history.
  \item The recommendation engine ranks candidate songs using a mood-aware scoring function and returns a playlist.
  \item The user plays songs and provides like/skip feedback, which is stored with the active mood label.
\end{enumerate}

\subsection{New/Extended Components}
\subsubsection{Client Application}
\begin{itemize}
  \item Adds a mood selector UI element on the home screen.
  \item Displays a “Mood Playlist” section populated by mood-aware recommendations.
\end{itemize}

\subsubsection{Backend Services}
\begin{itemize}
  \item Extends the recommendation API to accept an optional mood parameter.
  \item Adds a mood preference handler that records the user’s selected mood.
\end{itemize}

\subsubsection{Recommendation Engine}
\begin{itemize}
  \item Incorporates mood metadata (e.g., mood labels, energy, valence) into the ranking algorithm.[web:13][web:18]
  \item Applies filters and boosts for songs whose mood attributes align with the selected mood.
\end{itemize}

\section{Database Design Updates}
\subsection{New/Modified Tables}
\begin{itemize}
  \item MoodPreference(userId, mood, selectedAt).
  \item Optional mood-related attributes added to Track, such as moodLabel or energy/valence fields.
  \item Interaction table extended with an optional mood column to record the active mood at the time of interaction.
\end{itemize}

\subsection{Example Relationships}
\begin{itemize}
  \item One User to many MoodPreference records.
  \item One Track may be associated with one or more mood labels via attributes or auxiliary tables.
\end{itemize}

\section{Interface Design}
\subsection{API Changes}
\begin{itemize}
  \item GET /songs/recommended?mood=\{mood\}: returns a mood-aware recommendation list.
  \item POST /moods/select: stores the user’s selected mood for the current session.
\end{itemize}

\section{Design Considerations}
\begin{itemize}
  \item Mood-based logic is designed to be modular, allowing additional moods or a more advanced mood detection model in future snapshots.[web:16]
  \item Default behavior falls back to Snapshot 1 recommendations when no mood is selected.
\end{itemize}

\end{document}
