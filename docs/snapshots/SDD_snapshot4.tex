\documentclass[12pt]{article}
\usepackage[margin=1in]{geometry}
\usepackage{longtable}
\usepackage{hyperref}

\title{Software Design Document (SDD)\\
AI Music Recommender System\\
Snapshot 4 - Final Release and Future Work}
\author{Project Group 5}
\date{\today}

\begin{document}
\maketitle

\section*{Version Description}
\begin{longtable}{p{2cm}p{9cm}p{3cm}}
\textbf{Version} & \textbf{Description} & \textbf{Date}\\ \hline
4.0 & Final SDD for Snapshot 4. Adds Quick Mix flow, performance refinements, and design notes for future work. & \today\\
\end{longtable}

\section{Introduction}
\subsection{Purpose}
This document refines the design for Snapshot 4 by adding a Quick Mix feature, performance-related improvements, and architectural notes that support future enhancements.

\subsection{Scope}
Snapshot 4 does not introduce a major new subsystem but extends existing modules and clarifies design decisions for maintainability and future growth.

\section{Architecture Updates}
\subsection{Quick Mix Workflow}
\begin{enumerate}
  \item The user triggers Quick Mix from the main screen.
  \item The client calls a Quick Mix endpoint with the user identifier and context (e.g., current mood, recent interactions).
  \item The backend collects candidate tracks from content-based, mood-based, and collaborative sources.
  \item The recommendation engine combines these sources into a hybrid ranked playlist.[web:14][web:19]
  \item The backend returns the playlist, and the client starts playback.
\end{enumerate}

\subsection{Caching and Optimization (Optional)}
\begin{itemize}
  \item Frequently requested recommendation lists (e.g., top mood playlists) may be cached briefly to reduce load.
  \item Database indexes may be added on key fields (userId, trackId, actionType, mood, follow relationships) to improve query performance.
\end{itemize}

\section{Interface Design}
\subsection{API Additions}
\begin{itemize}
  \item GET /songs/quick-mix: returns a hybrid playlist combining multiple recommendation strategies.
  \item Optional: endpoints to expose diagnostic data for debugging recommendation issues (for internal use only).
\end{itemize}

\section{Design Considerations and Future Work}
\subsection{Design Decisions}
\begin{itemize}
  \item The system uses a hybrid design combining content-based, mood-based, and collaborative filtering to leverage strengths of different recommendation techniques.[web:14][web:17]
  \item Separation of concerns is maintained through distinct modules for recommendation, mood handling, and social features.
\end{itemize}

\subsection{Future Work Support}
\begin{itemize}
  \item The architecture is designed to accommodate more advanced ML models and context-aware features without major structural changes.[web:13][web:16]
  \item Additional services can be added (e.g., analytics, explanation service) using the same API-driven modular approach.
\end{itemize}

\end{document}
