\documentclass[12pt]{article}
\usepackage[margin=1in]{geometry}
\usepackage{longtable}
\usepackage{hyperref}

\title{Software Design Document (SDD)\\
AI Music Recommender System\\
Snapshot 1 - Core System}
\author{Project Group 5}
\date{\today}

\begin{document}
\maketitle

\section*{Version Description}
\begin{longtable}{p{2cm}p{9cm}p{3cm}}
\textbf{Version} & \textbf{Description} & \textbf{Date}\\ \hline
1.0 & Initial SDD for Snapshot 1 of the AI Music Recommender System. Describes base architecture, core components, and database design. & \today\\
\end{longtable}

\section{Introduction}
\subsection{Purpose}
This document describes the internal design and architecture for Snapshot 1 of the AI Music Recommender System, mapping the SRS requirements to implementable components.

\subsection{Scope}
Snapshot 1 includes a client interface, backend services, a simple recommendation component, and a relational database for user, catalog, and interaction data.

\subsection{Intended Audience}
The audience includes project team members, the course instructor, teaching assistants, and future developers who need to understand the initial design.

\section{System Overview}
The system is composed of:
\begin{itemize}
  \item A client application for user interaction.
  \item Backend services providing REST APIs for authentication, catalog, recommendations, and interaction logging.
  \item A recommendation engine using past interactions for basic personalization.
  \item A database layer storing users, songs, artists, and interactions.
\end{itemize}

\section{Architecture Design}
\subsection{Workflow}
\begin{enumerate}
  \item The user opens the client and logs in.
  \item The client sends credentials to the authentication API.
  \item On success, the client requests recommended songs for the user.
  \item The backend reads interaction history, runs recommendation logic, and returns a ranked list of tracks.
  \item The user plays songs and can like or skip them.
  \item The client logs play, like, and skip events via the interaction API, which stores them in the database.
\end{enumerate}

\subsection{Main Components}
\subsubsection{Client Application}
\begin{itemize}
  \item Login/registration screens.
  \item Home screen displaying recommended songs.
  \item Search screen.
  \item Now-playing screen with playback and feedback controls.
\end{itemize}

\subsubsection{Backend Services}
\begin{itemize}
  \item Authentication module (register, login).
  \item Catalog module (list songs, search songs).
  \item Recommendation module (fetch recommendations).
  \item Interaction logger (store plays, likes, skips).
\end{itemize}

\subsubsection{Recommendation Engine}
\begin{itemize}
  \item Reads the user’s interaction history.
  \item Computes simple scores based on likes and skips.
  \item Produces a ranked list of candidate tracks for the backend.
\end{itemize}

\section{Database Design}
\subsection{Schema}
Entities for Snapshot 1:
\begin{itemize}
  \item User(userId, name, email, passwordHash, createdAt).
  \item Artist(artistId, name).
  \item Track(trackId, title, artistId, album, genre, duration).
  \item Interaction(interactionId, userId, trackId, actionType, timestamp).
  \item Playlist(optional)(playlistId, userId, name).
\end{itemize}

\subsection{Relationships}
\begin{itemize}
  \item One User to many Interactions.
  \item One Track to many Interactions.
  \item One Artist to many Tracks.
  \item One User to zero or many Playlists.
\end{itemize}

\section{Interface Design}
\subsection{API Endpoints (Logical)}
Example logical endpoints:
\begin{itemize}
  \item POST /auth/register
  \item POST /auth/login
  \item GET /songs/recommended
  \item GET /songs/search
  \item POST /interactions
\end{itemize}

\section{Design Constraints}
\begin{itemize}
  \item Design must be simple enough for implementation within the course schedule.
  \item Components should be modular to support additional features in later snapshots.
\end{itemize}

\section{Glossary}
\begin{itemize}
  \item UI: User Interface.
  \item API: Application Programming Interface.
  \item DB: Database.
  \item SRS: Software Requirements Specification.
  \item SDD: Software Design Document.
\end{itemize}

\end{document}
